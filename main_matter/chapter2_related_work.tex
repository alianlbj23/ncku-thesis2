\chaptertitle{Chapter 2}{Related Work}  % "Chapter 1" 和 "Introduction" 套用格式
\section{數位孿生起源}
數位孿生的概念最早可追溯至 2002 年,由美國密西根大學的 Michael Grieves 教授\cite{Grieves2015DigitalTwin}在產品生命周期管理 (PLM, Product Lifecycle Management) 領域提出,
當時, Grieves 教授在密西根大學的產品生命週期管理 (PLM) 高階主管課程中首次引入了這個想法。然而, Grieves 教授也指出,在 2003 年,實際產品的數位化相對較新且處於不成熟的
階段。此外,當時關於物理產品在生產過程中資訊的收集也相當有限,主要依賴人工且多為紙本記錄。
儘管如此, Grieves 教授的前瞻性概念為未來的發展奠定了基礎。在隨後的十年中,支援虛擬產品開發與維護,
以及實體產品設計與製造的資訊科技經歷了顯著的進步。虛擬產品變得更加豐富,能夠更精確地代表其物理對應物。同時,
工廠車間中製造執行系統 (MES) 的普及,以及包括感測器、量測儀器、三次元量床、雷射、視覺系統和白光掃描等非破壞性感測技術的廣泛應用,
使得關於物理產品生產和形式的數據收集從手動紙本轉變為數位化。這些技術的進步為數位雙生概念的發展和應用提供了重要的支撐。
Grieves 等人\cite{Grieves2017DigitalTwin}提出了數位孿生的核心概念
旨在透過孿生環境中設計、測試、模擬來降低實體系統在現實中運作的風險並分析, 透過預先在虛擬環境進行開發和模擬
可以減少對昂貴且耗時的實體機器依賴也同時預測了該機器預期的表現, 此外數位孿生環境也為新開發機具的訓練與測試提供
了一個安全且高效的平台工程師和操作人員可以在不影響實際生產線的情況下熟悉設備的操作流程和潛在問題
雖然 Grieves 教授提出了理論,但由於當時感測器技術與資料處理能力有限,無法將其應用於大規模的場域。
\section{數位孿生應用趨勢}
將該技術真正的實際應用始於美國太空總署 (NASA) 和美國空軍的研究機構中, Stargel \cite{Glaessgen2012DigitalTwin}於文內提出
未來航太載具的運行環境將更加嚴苛, 傳統的設計與測試方法已無法滿足需求。因此, 透過數位孿生技術, NASA 能夠整合超高精度模擬、
健康管理系統(Integrated Vehicle Health Management, IVHM)以及歷史數據,以即時監測與預測航太載具的狀態,而美國空軍研究實驗室 (AFRL) 亦將數位孿生技術應用於軍用飛
機的結構維護與健康監測, Eric J. Tuegel 等人在Airframe Digital Twin 研究中提出傳統飛機結構壽命的預測方式主要依賴於經驗和定期檢測,然而,這些方法往往無法準確反映個別
飛機的真實狀況。為了提高預測準確度美國空軍引入數位孿生技術,透過整合機載感測器 (Structural Integrity Monitoring) ,以及有限元素分析 (FEA) 建立一個可即時更新的數位孿生模型。
在 Airframe Digital Twin 框架下,飛機的結構健康狀態將透過感測器資料即時更新, 使用該資料預測壽命, 傳統的飛機維護僅能透過每架飛機的飛行時數和使用週期來做判斷, 但透過數位孿生技術
則能根據每架飛機的的實際使用情況動態調整維修策略,從而大幅減少維護成本並提升機體壽命,此外,數位孿生還能夠模擬各種條件下的飛機結構性能,從而提供更多的設計參數供工程師參考。
該研究結果顯示透過數位孿生技術,飛機結構壽命的預測誤差可顯著降低並能夠提前預警潛在的結構故障。這項技術不僅提升了美國空軍的預防性維護(Predictive Maintenance) 能力,也為未來無人機(UAVs)
與新型戰機的長期運行提供了更精確的健康管理方案。
關於數位孿生的理解存在著不同的觀點,一些研究人員認為數位孿生研究應著重於模擬,而另一派認為數位孿生應該是一個全面的概念,包括模擬、資料分析,基於此論點,Tao等人進一步提出了應包含
物理、虛擬、連接、資料、服務五個維度。意味著期基礎理論來自了各種不同的領域,例如資訊工程、
\section{數位孿生關鍵技術}
\section{數位孿生應用領域}
\section{數位孿生環境}
近年來數位孿生 (Digital Twin)技術已獲得廣泛的關注其概念不僅限於學術研究更在工業界展現出巨大的應用潛力。
許多工廠開始積極導入數位孿生技術以期在虛擬環境中預測和避免實體機器的故障從而降低營運風險和維護成本。
過去僅能使用實體機器進行實驗, 往往在實驗過程中會發現許多無法預測的狀況
而造成機器損壞甚至對使用者造成傷害, 。\\
另一方面
數位孿生也成為強化機器學習(Machine Learning)及強化學習(Reinforcement Learning, RL)模型訓練的重要工具。
透過在高度擬真的數位環境中進行大量的模擬可以產生豐富的訓練數據加速 RL 模型的學習過程並驗證其在各種情境下的性能.
Zhou等人\cite{DigitalTwin}針對利用遊戲引擎構建數位孿生環境,
提出了一種基於 Unity 遊戲引擎的數位孿生模擬方法該研究不僅闡述了數位孿生的基本定義即物理世界實體的數位複製品,
更進一步提出了包含物理世界與數位世界的雙層架構. 在此架構下數位世界被細分為三個關鍵層次:利用 Unity 遊戲物件模擬「硬體」、
運用 Unity Scripting API 模擬「軟體」、以及整合外部工具以擴展模擬功能. 文獻中強調Unity 作為一個強大的遊戲引擎其在圖形渲染、
物理引擎和場景管理方面的優勢使其非常適合構建高擬真度的數位孿生環境. 此外Unity易於使用的介面和豐富的資產商店 也大大簡化了開發流程. \\
在模擬機器人動力學方面, Unity的Articulationbody\cite{UnityArticulation2024}提供了一系列的關節動力學模擬的工具, 讓開發者能夠輕鬆
的建立並調整結構, 內建對於剛體連接的控制, 有效的模擬關節之間的扭距, 摩擦力與阻尼, 並支援精準的碰撞與物理互動, 從而實現高度真實的機器人模擬.
\section{現有研究挑戰}
\section{本研究所提出的創新觀點}