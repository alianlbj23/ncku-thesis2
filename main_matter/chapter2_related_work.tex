\chaptertitle{Chapter 2}{Related Work}  % "Chapter 1" 和 "Introduction" 套用格式
\section{數位孿生起源}
數位孿生的概念最早可追溯至 2002 年,由美國密西根大學的 Michael Grieves 教授\cite{Grieves2015DigitalTwin}在產品生命周期管理 (PLM, Product Lifecycle Management) 領域提出,
當時, Grieves 教授在密西根大學的產品生命週期管理 (PLM) 高階主管課程中首次引入了這個想法。然而, Grieves 教授也指出,在 2003 年,實際產品的數位化相對較新且處於不成熟的
階段。此外,當時關於物理產品在生產過程中資訊的收集也相當有限,主要依賴人工且多為紙本記錄。
儘管如此, Grieves 教授的前瞻性概念為未來的發展奠定了基礎。在隨後的十年中,支援虛擬產品開發與維護,
以及實體產品設計與製造的資訊科技經歷了顯著的進步。虛擬產品變得更加豐富,能夠更精確地代表其物理對應物。同時,
工廠車間中製造執行系統 (MES) 的普及,以及包括感測器、量測儀器、三次元量床、雷射、視覺系統和白光掃描等非破壞性感測技術的廣泛應用,
使得關於物理產品生產和形式的數據收集從手動紙本轉變為數位化。這些技術的進步為數位雙生概念的發展和應用提供了重要的支撐。
Grieves 等人\cite{Grieves2017DigitalTwin}提出了數位孿生的核心概念
旨在透過孿生環境中設計、測試、模擬來降低實體系統在現實中運作的風險並分析, 透過預先在虛擬環境進行開發和模擬
可以減少對昂貴且耗時的實體機器依賴也同時預測了該機器預期的表現, 此外數位孿生環境也為新開發機具的訓練與測試提供
了一個安全且高效的平台工程師和操作人員可以在不影響實際生產線的情況下熟悉設備的操作流程和潛在問題
雖然 Grieves 教授提出了理論,但由於當時感測器技術與資料處理能力有限,無法將其應用於大規模的場域。
\section{數位孿生應用趨勢}
將該技術真正的實際應用始於美國太空總署 (NASA) 和美國空軍的研究機構中, Stargel \cite{Glaessgen2012DigitalTwin}於文內提出
未來航太載具的運行環境將更加嚴苛, 傳統的設計與測試方法已無法滿足需求。因此, 透過數位孿生技術, NASA 能夠整合超高精度模擬、
健康管理系統(Integrated Vehicle Health Management, IVHM)以及歷史數據,以即時監測與預測航太載具的狀態,而美國空軍研究實驗室 (AFRL) 亦將數位孿生技術應用於軍用飛
機的結構維護與健康監測, Eric J. Tuegel \cite{Tuegel2011DigitalTwin}等人在Airframe Digital Twin 研究中提出傳統飛機結構壽命的預測方式主要依賴於經驗和定期檢測,然而,這些方法往往無法準確反映個別
飛機的真實狀況。為了提高預測準確度美國空軍引入數位孿生技術,透過整合機載感測器 (Structural Integrity Monitoring) ,以及有限元素分析 (FEA) 建立一個可即時更新的數位孿生模型。
在 Airframe Digital Twin 框架下,飛機的結構健康狀態將透過感測器資料即時更新, 使用該資料預測壽命, 傳統的飛機維護僅能透過每架飛機的飛行時數和使用週期來做判斷, 但透過數位孿生技術
則能根據每架飛機的的實際使用情況動態調整維修策略,從而大幅減少維護成本並提升機體壽命,此外,數位孿生還能夠模擬各種條件下的飛機結構性能,從而提供更多的設計參數供工程師參考。
該研究結果顯示透過數位孿生技術,飛機結構壽命的預測誤差可顯著降低並能夠提前預警潛在的結構故障。這項技術不僅提升了美國空軍的預防性維護(Predictive Maintenance) 能力,也為未來無人機(UAVs)
與新型戰機的長期運行提供了更精確的健康管理方案。
關於數位孿生的理解存在著不同的觀點,一些研究人員認為數位孿生研究應著重於模擬,而另一派認為數位孿生應該是一個全面的概念,包括模擬、資料分析,基於此論點,Tao等人\cite{Tao2019DigitalTwin}進一步提出了應包含
物理、虛擬、連接、資料、服務五個維度。
\begin{itemize}
    \item 物理部分 (Physical Part) :
          基於生產工程的知識體系,主要描述對於真實物理的關鍵特徵以及其基礎理論,
          涵蓋了對真實產品、設備、系統的理解,是建立數位孿生不可或缺的基礎。
    \item 虛擬部分 (Virtual Part) :
          建立於製造工程以及電腦科學相關理論,製造工程提供了物理實體特性和行為映射到虛擬空間的理論,
          而電腦科學則是提供建模和模擬的工具和技術,例如幾何建模,物理模擬等等,虛擬模型需要能夠在虛擬空間
          重現其真實物理相同特徵和行為。
    \item 連接 (Connection) :
          涉及到虛擬、物理部分之間的連接,主要包括了數據的收集、傳輸、處理等技術,
          以及虛擬與物理部分之間的數據同步,以資訊科學的理論為基礎,進行資料傳輸,資料格式,和資料來源保護,
          而通訊科學提供了連接所需的網路技術和通協議。
    \item 數據 (Data) :
          此為資訊科學的核心,數位孿生依賴於物理世界所收集的大量資料,包刮物理參數、虛擬參數以及他們之間的資料融合,
          資訊科學定義了資料格式定義、操作流程、資料儲存、資料預處理、資料探勘、資料最佳化等理論及技術,以便從原始資料中獲取有用的資訊。
    \item 服務 (Service) :
          提供了服務的封裝、測試、部署、運行、維護等技術,以及服務的管理和整合等,讓製造工程能根據分析結果提供具體的服務。
\end{itemize}
內文提到數位孿生的五維模型是多項科學的交叉產物,它整合了資訊科學處理資料和服務的能力,製造工程對物理實體和生產過程的理解,
資料科學從資料中提取有利資訊,以及電腦科學提供建模、模擬和系統實現的工具。這種跨學科的特性使得數位孿生能夠實現物理空間和網路空間
的深度融合,從而在智慧製造和工業 4.0 中發揮關鍵作用,以下公司已將數位孿生技術應用於其中 : \\
\begin{itemize}
    \item 西門子 (Siemens) :
          西門子公司在數位孿生領域的研究已經取得了一定的成果,將其應用於電力系統與廢水處理廠,例如
          為芬蘭的電力系統開發孿生系統以提高自動化和能源決策,以及開發廢水處理廠的孿生環境以進行及時管道監控、節省能源和故障預測
    \item 通用電器 (General Electric, GE) :
          通用電器公司擁有四項數位孿生專利,其中包括了機械設備的數位孿生技術,
          以及用於工業設備的數位孿生技術,通用電器公司將數位孿生技術應用於風力發電機的設計與維護,以及石油和天然氣行業的設備監控並證明
          可改變系統開發、營運、維護模式且可提高效率高達 20\%,也將該技術應用於機車的生命週期追蹤、醫療健保等醫院最佳化管理。
    \item 英國石油 (British Petroleum, BP) :
          該公司將此技術應用於監控和維護偏遠地區的石油氣體設備,例如在阿拉斯加中的一個石油探勘和生產設施內部屬孿生系統以提高可行行性。
    \item 空中巴士 (Airbus) :
          其目標是透過孿生環境實現工廠的數位化,應用於監控製造過程並最佳化營運模式。
    \item 國際商業機器公司 (International Business Machines Corporation, IBM) :
          將孿生環境應用於自動駕駛車輛以分析引擎轉速、油壓和其他關鍵參數,不僅能有效的預防故障,還能開發更高效的引擎。
\end{itemize}
在城市開發,Wang 等人\cite{Glaessgen2012DigitalTwin}文中先定義了數位孿生的定義,透過即時數據更新,並利用模擬、機器學習來推理及輔助
政策,同時,也闡述了智慧城市的定義,強調其利用資訊與通訊科技(ICT)來感知、分析和整合核心系統的關鍵資訊,以智慧地應對各種需求。\\
內文指出城市是多種即時更新數據和資訊的載體,需要一個能夠即時獲取和管理這些數據的系統,數位孿生技術由於其整合物理和虛擬數據、進行即時分析和
預測的能力,被認為能更好地管理智慧城市,透過創建虛擬的「測試模型」,數位孿生可以幫助城市開發者主動測試不同的情境,文獻將數位雙生在智慧城市中
的應用根據數據生命週期管理過程(收集、儲存、建模、可視化、連接和利用)進行分類 :
\begin{itemize}
    \item 數據收集與儲存 :
          討論了各種感測器(GPS、IMU、無線感測器)、遙控感應、建築資訊模型(BIM)和地理資訊系統(GIS)等技術的應用。
    \item 數據建模與可視化 :
          重點介紹了 BIM、城市資訊模型(CIM)、擴增實境(AR)和虛擬實境(VR)等技術如何用於數位化表示城市的不同組件。
    \item 資料連接 :
          探討了數據標準格式、5G、邊緣計算、區塊鏈和物聯網(IoT)等技術在實現不同系統之間互操作性和高效數據共享方面的作用。
    \item 數據利用 :
          闡述了如何利用收集到的數據進行分析、模擬,以支援交通、環境、能源、醫療保健、安全和教育等領域的決策制定。
\end{itemize}
Wang 等人\cite{White2021DigitalTwin}目標是建構一個開放且可互動的數位孿生環境,並且強調能讓市民能夠透過 3D 虛擬城市與政府機關互動提供城市規劃反饋,
文中提到該模型由 6 個部分組成,分別是地形、建築、基礎設施、移動性、數位層和虛擬層。\\
主要實驗有以下部分 :
\begin{itemize}
    \item Skyline Simulation and Public Participation :
          在都市發展中,建築物的高度與密度可能影響市容景觀與日照條件,因此天際線模擬(Skyline Simulation) 是都市規劃的重要工具。
          以往的研究使用 BIM(Building Information Modeling) 來預測新建築對城市環境的影響,但市民參與度仍然有限。一些研究已嘗試將 3D 模型開放
          給市民,以提高都市規劃透明度。然而,目前的研究多為靜態模型,缺乏動態回饋機制。該目標透過數位孿生建立即時互動平台,允許市民觀看新建築提案
          ,並提供意見,進一步提升市民參與度。
    \item Green Space and Urban Planning
          綠地與公園對於提升城市居民的生活品質至關重要。研究顯示,城市綠地可以降低熱島效應、
          改善空氣品質並促進社會互動)。然而,目前的都市規劃系統主要依賴於靜態 GIS 數據,而忽略了
          市民對綠地需求的回饋。該研究透過數位孿生模擬最佳的綠地設計,並允許市民提交建議(如增加健身器材或兒童遊樂場
          ),從而動態調整城市規劃。
    \item Flooding Simulation and Disaster Preparedness\
          洪水預測與防災規劃在智慧城市研究中占據重要地位。目前已有許多研究利用數值模型(如 MIKE Flood)
          來模擬城市洪水風險,並結合 GIS 數據來標記高風險區域。然而,這些傳統方法難以即時更新,
          且缺乏市民參與。該研究透過數位孿生建立即時洪水模擬系統,使都市規劃者能夠測試最佳的疏散路線與防洪策略,
          並讓市民直接參與風險評估,提高城市的防災能力。
    \item Crowd Simulation and Pedestrian Behavior
          人群行為模擬(Crowd Simulation)在都市規劃與緊急應變管理中具有重要應用價值。
          目前的研究已使用 AI 與機器學習技術來模擬人流模式,並考慮不同年齡層的移動習慣
          現有的研究多為靜態分析,未能結合即時數據與市民回饋。本研究整合數位孿生與人群模擬,提供更準確的行人流動預測,並評估不同人群對城市設施的影響。
\end{itemize}
根據以上的研究,數位孿生技術在智慧城市的應用已經取得了一定的成果,除了在工業製造上有一定的成果外,也在城市規劃、環境保護等領域有著一席之地。
\section{現有孿生環境探討}
\subsection{Gazebo}
Koenig 等人製作出了一款開源的機器人模擬器 Gazebo\cite{Koenig2004Gazebo},其目標旨在快速測試新概念、策略與演算法,在以前的模擬器僅支援 2D 環境,無法滿足
現今的機器人實驗且缺乏高擬真的物理模擬,無法模擬出外在環境對於機器人的影響,對此 gazebo 提供了 3D 動態模擬環境,使其具備真實物理模擬能力,而其開源的特性
使得研究人員可自由修改與擴充功能,該模擬器的核心架構以下幾點:
\begin{itemize}
    \item 高擬真 3D 物理模擬
          \subitem 剛體動力學 (Rigid Body Dynamics) : 所有物體均具質量、摩擦力、速度等屬性,可進行物理互動(如推、拉、翻滾等)。
          \subitem 關節動力學 (Joint Dynamics) : 支援多種關節類型,如旋轉、滑動、球形等,可模擬機器人的關節運動。
          \subitem 碰撞檢測 (Collision Detection) : 可檢測物體間的碰撞,並計算碰撞後的反應,如彈跳、摩擦力等。
          \subitem 重力及環境模擬 (Gravity Simulation) : 可模擬物體受到重力的影響,並計算物體的運動軌跡和模擬不同地形、摩擦係數與障礙物影響。
    \item 物理引擎與視覺化
          \subitem ODE 物理引擎 (ODE Physics Engine) : 採用 Open Dynamics Engine (ODE) 進行剛體模擬,具備碰撞檢測、摩擦力模擬、重量計算等功能。
          \subitem OpenGL 視覺化 (OpenGL Visualization) : 使用 OpenGL 進行 3D 圖形渲染,提供高品質的視覺化效果。
          \subitem ROS 整合 (ROS Integration) : 支援 ROS(Robot Operating System)通訊協議,可與 ROS 系統進行無縫整合。
          \subitem 感測器模擬 (Sensor Simulation) :
          可支援 lidar、RGBD camera、GPS、IMU 等感測器模擬,並提供感測器數據的模擬與輸出。
    \item 世界建模與擴展性
          \subitem 世界建模 (World Modeling) : 支援多種世界模型,如平面、立方體、球體等,可自定義世界模型的大小、形狀、材質等。
          \subitem 擴展性 (Extensibility) : 提供豐富的 libary,可擴展新的物理模型、感測器模型、控制器模型等,並支援多種程式語言。
          \subitem 地形模擬 (Terrain Simulation) : 支援地形模擬,可模擬不同地形對機器人運動的影響,如坡度、障礙物等。
    \item 新環境建模與導航測試
          \subitem 透過手繪地圖或衛星數據建構虛擬場景,測試 SLAM 演算法。
          \subitem 提前模擬光線變化、地形影響,提高機器人適應力。
\end{itemize}
縱使 gazebo 在模擬上已取得強大的優勢,但還是會有一些缺點如下 :
\begin{itemize}
    \item 物理模擬的不足 :
          由於 gazebo 使用 ODE 物理引擎,其模擬精度有限,無法模擬複雜的物理現象。
    \item 計算資源需求高 :
          gazebo 需要大量的計算資源,對硬體要求較高,並且在模擬大型場景時容易出現卡頓現象。
    \item 視覺渲染能力不足 :
          gazebo 使用 OpenGL 進行視覺化,其渲染效果較為簡單,無法達到高逼真度的效果。
    \item 缺乏 AI 與機器學習訓練工具 :
          若要訓練強化學習 (Reinforcement Learning, RL) 需額外搭配 Gym-Ignition 或 ROS2 + OpenAI Gym,
          整合較複雜且主要適用於 控制演算法測試 (如 SLAM、導航、運動控制),不適合用於大規模 AI 訓練。
    \item 版本支援問題 :
          gazebo 的版本更新較為頻繁,且不同版本之間的兼容性較差,使用者需要不斷更新並解決版本衝突問題且該模擬器主要
          支援 Linux,於 windows 的版本支援較差。
\end{itemize}
\subsection{Webots}
由 Cyberbotics Ltd. \cite{Michel2004Webots} 開發的機器人模擬器,由瑞士洛桑聯邦理工學院 (EPFL) 的 BIRG & SWIS 研究小組所開發,
最初由 1998 年發布並維護至今,支援跨平台運作,該模擬器有以下特點 :
\begin{itemize}
    \item 完整的機器人模擬支持 :
          裡面支援輪式、足式、飛行機器人,以及多機器人協作系統,並提供多種感測器模擬,如 lidar、RGBD camera、GPS、IMU 等,在控制的
          部分可支援 c、c++、python、java 等程式語言開發或透過 TCP/IP 連接到 MATLAB、ROS 等外部軟體。
    \item
\end{itemize}
\section{數位孿生應用領域}
\section{數位孿生環境}
近年來數位孿生 (Digital Twin)技術已獲得廣泛的關注其概念不僅限於學術研究更在工業界展現出巨大的應用潛力。
許多工廠開始積極導入數位孿生技術以期在虛擬環境中預測和避免實體機器的故障從而降低營運風險和維護成本。
過去僅能使用實體機器進行實驗, 往往在實驗過程中會發現許多無法預測的狀況
而造成機器損壞甚至對使用者造成傷害, 。\\
另一方面
數位孿生也成為強化機器學習(Machine Learning)及強化學習(Reinforcement Learning, RL)模型訓練的重要工具。
透過在高度擬真的數位環境中進行大量的模擬可以產生豐富的訓練數據加速 RL 模型的學習過程並驗證其在各種情境下的性能.
Zhou等人\cite{DigitalTwin}針對利用遊戲引擎構建數位孿生環境,
提出了一種基於 Unity 遊戲引擎的數位孿生模擬方法該研究不僅闡述了數位孿生的基本定義即物理世界實體的數位複製品,
更進一步提出了包含物理世界與數位世界的雙層架構. 在此架構下數位世界被細分為三個關鍵層次:利用 Unity 遊戲物件模擬「硬體」、
運用 Unity Scripting API 模擬「軟體」、以及整合外部工具以擴展模擬功能. 文獻中強調Unity 作為一個強大的遊戲引擎其在圖形渲染、
物理引擎和場景管理方面的優勢使其非常適合構建高擬真度的數位孿生環境. 此外Unity易於使用的介面和豐富的資產商店 也大大簡化了開發流程. \\
在模擬機器人動力學方面, Unity的Articulationbody\cite{UnityArticulation2024}提供了一系列的關節動力學模擬的工具, 讓開發者能夠輕鬆
的建立並調整結構, 內建對於剛體連接的控制, 有效的模擬關節之間的扭距, 摩擦力與阻尼, 並支援精準的碰撞與物理互動, 從而實現高度真實的機器人模擬.
\section{Yolo}

