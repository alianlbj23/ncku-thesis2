\chaptertitle{Chapter 2}{Related Work}  % "Chapter 1" 和 "Introduction" 套用格式
\section{數位孿生環境}
近年來數位孿生 (Digital Twin)技術已獲得廣泛的關注其概念不僅限於學術研究更在工業界展現出巨大的應用潛力。
許多工廠開始積極導入數位孿生技術以期在虛擬環境中預測和避免實體機器的故障從而降低營運風險和維護成本。
過去僅能使用實體機器進行實驗, 往往在實驗過程中會發現許多無法預測的狀況
而造成機器損壞甚至對使用者造成傷害, 對此Michael等人\cite{Grieves2017DigitalTwin}提出了數位孿生的核心概念
旨在透過孿生環境中設計、測試、模擬來降低實體系統在現實中運作的風險並分析, 透過預先在虛擬環境進行開發和模擬
可以減少對昂貴且耗時的實體機器依賴也同時預測了該機器預期的表現, 此外數位孿生環境也為新開發機具的訓練與測試提供
了一個安全且高效的平台工程師和操作人員可以在不影響實際生產線的情況下熟悉設備的操作流程和潛在問題。\\
另一方面
數位孿生也成為強化機器學習(Machine Learning)及強化學習(Reinforcement Learning, RL)模型訓練的重要工具。
透過在高度擬真的數位環境中進行大量的模擬可以產生豐富的訓練數據加速 RL 模型的學習過程並驗證其在各種情境下的性能.
Zhou等人\cite{DigitalTwin}針對利用遊戲引擎構建數位孿生環境,
提出了一種基於 Unity 遊戲引擎的數位孿生模擬方法該研究不僅闡述了數位孿生的基本定義即物理世界實體的數位複製品,
更進一步提出了包含物理世界與數位世界的雙層架構. 在此架構下數位世界被細分為三個關鍵層次:利用 Unity 遊戲物件模擬「硬體」、
運用 Unity Scripting API 模擬「軟體」、以及整合外部工具以擴展模擬功能. 文獻中強調Unity 作為一個強大的遊戲引擎其在圖形渲染、
物理引擎和場景管理方面的優勢使其非常適合構建高擬真度的數位孿生環境. 此外Unity易於使用的介面和豐富的資產商店 也大大簡化了開發流程. \\
在模擬機器人動力學方面, Unity的Articulationbody\cite{UnityArticulation2024}提供了一系列的關節動力學模擬的工具, 讓開發者能夠輕鬆
的建立並調整結構, 內建對於剛體連接的控制, 有效的模擬關節之間的扭距, 摩擦力與阻尼, 並支援精準的碰撞與物理互動, 從而實現高度真實的機器人模擬.
