\documentclass[a4paper,12pt]{report}

\usepackage{fontspec}
\usepackage{xeCJK}

% 設定中文字型(Windows: SimSun,macOS/Linux: Noto Sans CJK SC)
\setmainfont{Times New Roman} % 英文字體
\setCJKmainfont{標楷體} % Windows
%\setCJKmainfont{Noto Sans CJK SC} % macOS/Linux

\usepackage{geometry}
\geometry{top=23mm, bottom=35mm, left=30mm, right=25mm}

% ===== 定義封面 =====
\newgeometry{top=23mm, bottom=30mm, left=20mm, right=20mm} % 設定封面邊界 [1]
\begin{titlepage}
  \centering
  {\fontsize{36pt}{\baselineskip}\selectfont \textbf{國立成功大學}} \par \vspace{1cm} % 校名 [1] (您可以調整字體大小)
  {\fontsize{24pt}{\baselineskip}\selectfont \textbf{您的系(所、學位學程)別}} \par \vspace{2cm} % 系(所、學位學程)別 [1] (請替換) (您可以調整字體大小)
  {\fontsize{48pt}{\baselineskip}\selectfont \textbf{您的論文標題 (中文)}} \par \vspace{0.5cm} % 論文名稱(中文) [1] (請替換) (您可以調整字體大小)
  {\fontsize{36pt}{\baselineskip}\selectfont Your Thesis Title (English)} \par \vspace{2cm} % 題目英文名稱 [1] (請替換) (您可以調整字體大小)
  {\fontsize{24pt}{\baselineskip}\selectfont 您的研究生姓名} \par \vspace{0.5cm} % 研究生姓名 [1] (請替換) (您可以調整字體大小)
  {\fontsize{20pt}{\baselineskip}\selectfont 指導教授姓名} \par \vspace{2cm} % 指導教授姓名 [1] (請替換) (您可以調整字體大小)
  {\fontsize{20pt}{\baselineskip}\selectfont 年份 月份 (學位考試通過日期)} \par % 年、月(學位考試通過日期) [1] (請替換) (您可以調整字體大小)
\end{titlepage}
\restoregeometry % 恢復內頁邊界 [1]

\begin{document}

% ===== 中文摘要 =====
\chapter*{\centering 中文摘要}
\begin{center}
  {\Large \textbf{論文標題 (請替換)}} % 修改你的論文標題
\end{center}
\vspace{1cm}

這裡是中文摘要內容,請填入摘要大綱。摘要應概述研究背景、研究目標、研究方法,以及主要結果和結論。

\vspace{1cm}

\noindent \textbf{關鍵字:} 這裡填入關鍵字,最多五個,請用逗號分隔。

\newpage

% ===== 英文摘要 (Extended Abstract) =====
\chapter*{\centering Extended Abstract}
\begin{center}
  {\Large \textbf{Thesis Title (Replace with Your Title)}} % 修改你的英文標題
\end{center}
\vspace{1cm}

This is the extended abstract, which should summarize the background, research objectives, methodology, key findings, and conclusions.

\vspace{1cm}

\noindent \textbf{Keywords:} Enter up to five keywords, separated by commas.

\end{document}